\newpage

    \begin{flushleft}
    \Large \textbf{Streszczenie}
    \end{flushleft}

    \vspace{1cm}
        W niniejszej pracy dyplomowej przedstawiam proces tworzenia w języku Kotlin aplikacji mobilnej na system operacyjny Android, na przykładzie
        aplikacji mobilnej dającej użytkownikom możliwość zapisu miejsc na mapie Wrocławia oraz planowania wycieczek z wykorzystaniem uprzednio zapisanych lokalizacji.
        Użytkownik może zapisywać miejsca i wycieczki jako prywatne (widoczne tylko dla niego) lub publiczne (wyświetlane na mapach innych użytkowników). \\ 
        
        
        \noindent Zgodnie z założeniami, aplikacja ma umożliwić jej użytkownikom:
        \begin{itemize}
            \item wyświetlanie swojej pozycji na mapie Wrocławia (tzw.\ lokalizacja);
            \item dodawanie Miejsc, którym można przypisać nazwę, opis oraz kategorię;
            \item tworzenie Wycieczek składających się z maksymalnie dziesięciu Miejsc;
            \item dodawanie Etykiet do wszystkich Miejsc publicznych; 
            \item rysowanie na mapie ścieżek prowadzących do Miejsc oraz obrazowanie trasy wycieczek. 
        \end{itemize}

        Aplikacja ma służyć jako narzędzie społecznościowe umożliwiające użytkownikom informowanie innych o ciekawych miejscach oraz tworzenie planów wycieczek.
        Użytkownicy mogą dodawać interesujące ich miejsca oraz wycieczki do list ,,Ulubionych''. 
        Użytkownik może wytyczyć ścieżkę do każdego Miejsca, a każdej
        Wycieczce towarzyszy możliwość wyświetlenia na mapie Wrocławia trasy przechodzącej przez wszystkie jej Miejsca w kolejności zadanej przez użytkownika. 

        Do przechowywania dane o użytkownikach, zdefiniowanych przez nich Miejscach oraz Wycieczkach w chmurze obliczeniowej Google Cloud moja aplikacja wykorzystuje platformę Firebase.
        Komunikacja z bazą danych zachodzi za pomocą bibliotek i API tworzonych przez firmę Google z myślą o aplikacjach mobilnych i webowych.

        Interfejs użytkownika zaprojektowany został z użyciem stworzonego przez Google stylu graficznego i języka projektowego Material Design.

\newpage

    \begin{flushleft}
    \Large \textbf{Abstract}
    \end{flushleft}

    \vspace{1cm}
        This thesis describes the process of creating a mobile application for Android operating system, written in the language Kotlin, on the example
        of a mobile application that allows its users to save places on the map of Wroclaw and to plan trips using the aforementioned places.
        The user can save places as private or public, which can be seen on the map by other users.
        Main goals of the project were:
        \begin{itemize}
            \item Detecting the position of the user on the map of Wroclaw.
            \item Letting the user add a Place containing name, description and belonging to a category.
            \item Letting the user create Trips consisting of at most ten places.
            \item Letting the user add Tags to every public Place.
            \item Drawing the paths on the map that lead to a place or chart the path of a trip.
        \end{itemize}

        The application is meant as a social network, allowing the users to share interesting places and to save their trip plans.
        Users can save places and trips by adding them to lists of Favourites. It is possible to chart a path to every place. Every trip allows the user
        to chart a path through all of its places in an order arranged by the user. \\ 

        The application stores the data of its users and their places and trips in Google Cloud Computing Engine using the service of the Firebase platform.
        Communication with the database is facilitated by libraries and API's created by Google with mobile and web applications in mind. \\ 

        User interface was created using Material Design — graphic style and project language created and maintained by Google.