\newpage
\section{Prezentacja aplikacji}
%\vspace{1cm}
    Aplikację mobilną podzielić można na dwa główne widoki. Pierwszy, z którym zetknie się użytkownik, to ekran autoryzacji. Użytkownik może tam stworzyć nowe konto lub zalogować się
    na konto już istniejące. Drugi widok to ekran główny, na którym odbywają się wszystkie kluczowe aktywności związane z  dzieleniem się 
    ciekawymi miejscami i planami wycieczek.

%\vspace{1cm}
    \subsection{Ekrany autoryzacji użytkownika}
        \subsubsection{Ekran powitalny}
        Pierwszy ekran, z którym zetknie się nowy użytkownik, to ekran powitalny (Rysunek~\ref{welcome}). Na górze ekranu powitalnego widoczne jest zdjęcie panoramy
        Wrocławia widoczne z kościoła p.w. św. Elżbiety~\cite{RYNEK}. Poniżej znajduje się tekst powitalny. Na samym dole, w wygodnym  miejscu ułożone są przyciski wyboru akcji.
        Pierwszy z nich, zatytułowany ,,SIGN UP'' przekieruje użytkownika do okna rejestracji. Drugi z nich otworzy okno przeglądarki, w którym użytkownik będzie mógł zalogować się za pomocą
        konta Twitter. Na samym dole znajduje się widok tekstowy dający powracającemu użytkownikowi możliwość zalogowania się za pomocą adresu e-mail. Każda pomyślna próba autoryzacji
        zakończy się przeniesieniem użytkownika z aktywności autoryzacji do  aktywności głównej.

      %  \vspace{0.5cm}
        \begin{figure}[!ht]%
            \centering
            \includegraphics[scale=0.09]{src/app/welcome_fragment.png}
            \caption{Ekran powitalny aplikacji.\label{welcome}}
            \qquad
        \end{figure} 

        \subsubsection{Ekran rejestracji}
        Ekran rejestracji (Rysunek~\ref{register}) składa się przede wszystkim z pól tekstowych, do których użytkownik może wprowadzić pożądane informacje. Poprawność wprowadzanych danych jest
        w trakcie procesu rejestracji weryfikowana. Adres e-mail musi być odpowiednio sformatowany, pseudonim musi być unikatowy, a hasło musi spełniać odpowiednie kryteria bezpieczeństwa. Poziom bezpieczeństwa
        hasła ukazywany jest za pomocą paska postępu zaczerpniętego z biblioteki Material Design. Za pomocą kolorów i tekstu sygnalizuje on użytkownikowi, czy hasło spełnia odpowiednie warunki. Aby móc
        utworzyć konto, hasło musi spełniać warunki opisane w instrukcji znajdującej się u dołu ekranu. Na samym dole znajduje się przycisk ,,REGISTER''. Wciśnięcie go powoduje weryfikację danych. W przypadku,
        jeśli któreś z pól wypełnione jest niepoprawnie, wyświetlane jest powiadomienie, a kursor użytkownika automatycznie przemieszcza się w wymagające uwagi pole.

      %  \vspace{0.5cm}
        \begin{figure}[!ht]%
            \centering
            \begin{subfigure}[b]{0.38\textwidth}
                \centering
                \includegraphics[width=\textwidth]{src/app/register1.png}
                \caption{Pusty ekran rejestracji.\label{register1}}
            \end{subfigure}
            \hfill
            \begin{subfigure}[b]{0.38\textwidth}
                \centering
                \includegraphics[width=\textwidth]{src/app/register2.png}
                \caption{Uzupełniony ekran rejestracji.\label{register2}}
            \end{subfigure}
            \caption{Ekran rejestracji użytkownika.\label{register}}
            \qquad
        \end{figure} 

        \subsubsection{Ekran logowania za pomocą konta Twitter}
        Użytkownik może uzyskać dostęp do aplikacji, logując się za pomocą konta Twitter. Reprezentujący tę opcję przycisk przekieruje użytkownika do przeglądarki, gdzie poproszony on zostanie o autoryzację
        dostępu aplikacji do konta Twitter (Rysunek~\ref{twitter}). Obecnie popularne jest oferowanie w aplikacjach mobilnych i internetowych opcji logowania za pomocą strony trzeciej, ponieważ
        wielu użytkowników może nie chcieć tworzyć nowego konta w danym serwisie. Dodanie tego typu opcji niweluje ten problem, pozwalając użytkownikowi na korzystanie z konta istniejącego już w
        innym serwisie. Przy tego typu rejestracji pseudonim użytkownika z serwisu Twitter zostanie ustawiony jako pseudonim w aplikacji. Jeśli pseudonim ten jest już zajęty,  użytkownik zostanie
        poproszony o wybór nowego, unikalnego pseudonimu.

     %   \vspace{0.5cm}
        \begin{figure}[!ht]%
            \centering
            \includegraphics[scale=0.14]{src/app/twitter_login.png}
            \caption{Okno autoryzacji za pomocą konta Twitter.\label{twitter}}
            \qquad
        \end{figure} 