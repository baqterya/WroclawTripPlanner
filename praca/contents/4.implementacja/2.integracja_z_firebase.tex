\newpage
    \subsection{Integracja z Firebase}
    Tworzenie aplikacji korzystającej z funkcji oferowanych przez Firebase należy rozpocząć od utworzenia nowego projektu w witrynie internetowej Firebase. W trakcie tego procesu wyświetlane są instrukcje
    przedstawiające kroki, które należy podjąć, aby połączyć aplikację z projektem chmury obliczeniowej. Sprowadza się to przede wszystkim do umieszczenia w folderze zawierającym kod źródłowy, pliku
    \emph{google-services.json}, automatycznie wygenerowanego w trakcie procesu. Zawiera on w sobie informacje o projekcie oraz klucze API pozwalające na połączenie z chmurą obliczeniową. 
    W następnej kolejności należy do aplikacji mobilnej dołączyć bilbioteki, na które składa się Firebase SDK for Android. Odbywa się to za pomocą systemu budowania projektów Gradle (patrz Listing~\ref{gradle}).

    \vspace{1cm}
    \begin{lstlisting}[language=Kotlin, caption=Biblioteki Firebase SDK zaimplementowane w aplikacji, label=gradle]
//Firebase
implementation platform('com.google.firebase:firebase-bom:29.0.0')
implementation 'com.google.firebase:firebase-analytics'
implementation 'com.google.firebase:firebase-auth-ktx:21.0.2'
implementation 'com.google.firebase:firebase-firestore-ktx:24.0.2'
    \end{lstlisting}
    \vspace{1cm}

        \subsubsection{Implementacja Firebase Authentication}
        Pierwszą powiązaną z Firebase funkcjonalnością aplikacji, z którą zetknie się nowy użytkownik aplikacji, jest Firebase Authentication, odpowiadające za rejestrację oraz logowanie użytkowników.
        W aplikacji zaimplementowana została możliwość tworzenia konta za pomocą adresu e-mail oraz logowanie się za pośrednictwem konta na platformie Twitter. Z poziomu SDK dostęp do funkcji autoryzacyjnych
        otrzymujemy za pomocą obiektu \emph{Firebase.auth}. To na nim wywoływane są metody tworzące nowe konto użytkownika oraz pozwalające na zalogowanie. Na Listingu~\ref{auth} zobaczyć można przykład
        zastosowania Firebase SDK for Android do utworzenia konta użytkownika za pomocą danych (email, password, oraz username) wprowadzonych przez niego w polach tekstowych oraz metodę pozwalającą
        na zalogowanie za pomocą konta w serwisie Twitter. Kiedy rejestracja bądź logowanie za pomocą konta Twitter zakończy się pomyślnie, konto użytkownika zostanie zapisane w serwisie autoryzacyjnym
        Firebase dostępnym za pomocą konsoli Firebase. Dodatkowo dane użytkownika zostaną wraz z jego pseudonimem zapisane w bazie danych Firestore, aby ułatwić komunikację w bazie danych.
        Hasło użytkownika pozostaje dyskretne nawet dla twórcy aplikacji. Za bezpieczeństwo tajnych danych odpowiadają zabezpieczenia chmury obliczeniowej Google. Twórca projektu, w razie potrzeby
        może uzyskać zaszyfrowane wersje haseł, np.\' na potrzeby migracji danych na inny serwer.

\newpage
        \begin{lstlisting}[
            language=Kotlin, 
            caption=Utworzenie konta użytkownika za pomocą \emph{Firebase.auth}, 
            label=auth,
        ]
auth = Firebase.auth

/** 
* stworzenie konta za pomoca email'u i hasla wprowadzonego 
* przez uzytkownika do pol tekstowych 
**/
Firebase.auth.createUserWithEmailAndPassword(email, password)


/**
* kod odpowiedzialny za polaczenie z serwisem Twitter
* otwiera w oknie aplikacji okno przegladarki, w ktorej
* uzytkownik zostanie poproszony o autoryzacje za pomoca
* danych konta Twitter 
**/
val provider = OAuthProvider.newBuilder("twitter.com")

val pendingTaskResult = auth.pendingAuthResult
if (pendingTaskResult == null) {
    auth.startActivityForSignInWithProvider(
        requireActivity(), provider.build()
    )
}
        \end{lstlisting}
        \vspace{1cm}

        \subsubsection{Implementacja Firebase Cloud Firestore}
        Dostęp do funkcjonalności Cloud Firestore otrzymać można za pomocą obiektu \emph{Firebase.firestore}. Za jego pomocą możemy dokonywać zapytań do bazy danych i otrzymywać odpowiedzi zwrotne.
        Przykład zaobserwować można w Listingu~\ref{add_place}, który pokazuje, jak za pomocą Firebase SDK wykonujemy proste polecenie dodania nowego miejsca do kolekcji \emph{places}.

        \vspace{1cm}
        \begin{lstlisting}[
            language=Kotlin, 
            caption=Dodanie miejsca do bazy danych za pomocą \emph{Firebase.firestore}, 
            label=add_place,
        ]
db = Firebase.firestore

db.collection("places").add(newPlace)
        \end{lstlisting}
        \vspace{1cm}

        Jak zauważyć można na Listingu~\ref{add_place} do bazy danych dodany jest obiekt o nazwie \emph{newPlace}. Jest to obiekt klasy \emph{Place}, który w procesie dodawania do bazy danych konwertowany jest
        na analogiczny słownik danych, umieszczany w drzewie JSON bazy danych. Każdy typ danych wykorzystany w aplikacji posiada klasę w Kotlinie modelującą jego pola w bazie danych. Na przykładzie 
        Listingu~\ref{place_class} zaobserwować można jak modelowany, jest obiekt miejsca z poziomu aplikacji.

        \vspace{1cm}
        \begin{lstlisting}[
            language=Kotlin, 
            caption=Klasa danych Miejsca zmodelowana w języku Kotlin, 
            label=place_class,
        ]
/**
* Data class model of a Place object in the database
*
* @property placeFavUsersId: List of users that added 
* the place to their favourites
**/
data class Place(
    var placeId: String? = null,
    var placeName: String? = null,
    var placeGeoHash: String? = null,
    var placeLatitude: Double? = null,
    var placeLongitude: Double? = null,
    var placeDescription: String? = null,
    var placeOwnerId: String? = null,
    var placeOwnerName: String? = null,
    var placeCategories: ArrayList<String> = arrayListOf(),
    var placeTagList: ArrayList<Tag> = arrayListOf(),
    var placeFavUsersId: ArrayList<String> = arrayListOf(),
    var placeLikes: Int = 0,
    var placeIsPrivate: Boolean = false,
)
        \end{lstlisting}
        \vspace{1cm}

        Powyższa klasa danych zostaje przez Firebase SDK przekonwertowana do formy, którą zaobserwować można na Rysunku~\ref{firestore_screen}. W wypadku kiedy dane z bazy danych muszą zostać wykorzystane
        w aplikacji, można wykorzystać konwersję w drugą stronę, która możliwa jest za pomocą funkcji \emph{toObject\!()} dostępnej w Firebase SDK.\@ Przykład pozyskania danych z bazy, aby wykorzystać je w
        aplikacji, zaobserwować można na Listingu~\ref{firebase_get}. Obiekt JSON przekształcany jest tam do analogicznej mu klasy \emph{Place}, aby w przejrzysty sposób uzyskiwać dostęp do danych, tutaj
        pozycji geograficznej, danego miejsca i umiejscowić jego znacznik na mapie, zatytułowany nazwą miejsca. Klasa Miejsca musi posiadać takie same nazwy pól jak klucze słownika JSON odpowiadającego
        Miejscu w bazie danych.

\newpage
        \begin{lstlisting}[
            language=Kotlin, 
            caption=Fragment kodu pobierający i wykorzystujący miejsce znajdujące się w bazie danych, 
            label=firebase_get,
        ]
// pobranie miejsca z bazy danych
db.collection("places").document(placeId).get()
// dodanie funkcji nasluchujacej sukcesu operacji
.addOnSuccessListener {
    // konwersja do obiektu klasy Place
    val place = it.toObject(Place::class.java)!!
    // umiejscowienie na mapie pinezki wykorzystujac 
    // dane pobrane z obiektu klasy Place
    map.addMarker(
        MarkerOptions()
            .title(place.placeName)
            .position(
                LatLng(
                    place.placeLatitude!!, 
                    place.placeLongitude!!
                )
            )
            .icon(bitmapDescriptorFromVector(
                    requireContext(), R.drawable.ic_map_pin
                ))
    )?.showInfoWindow()
}
        \end{lstlisting}
        \vspace{1cm}

        Jak wspomniane zostało w rozdziale~\ref{struktura} aplikacja oddziela obsługę bazy danych od obsługi interfejsu korzystając ze wzorca projektowego MVVM.\@ W aplikacji utworzona została
        klasa FirestoreViewModel, która odpowiada za bezpośrednią komunikację z bazą danych, oddzielając ją tym samym od części obsługującej interfejs użytkownika. Dla przykładu w Listingu~\ref{viewmodel}
        zobaczyć możemy fragment kodu, który pobiera dane wprowadzone przez użytkownika w trakcie tworzenia nowej wycieczki, aby następnie przesłać je do obiektu FirestoreViewModel w celu wpisania wycieczki
        do bazy danych Firestore.

\newpage
        \begin{lstlisting}[
            language=Kotlin, 
            caption=Fragment kodu obsługujący interfejs użytkownika i komunikujący się z bazą danych za pomocą FirestoreViewModel, 
            label=viewmodel,
        ]
// deklaracja obiektu ViewModel
val firestoreViewModel = FirestoreViewModel()
// deklaracja obiektu reprezentujacego nowa wycieczke
val newTrip = Trip()

// program sprawdza, czy nazwa i opis wycieczki 
// zostaly wprowadzone przez uzytkownika
if (inputCheck(tripName) && inputCheck(tripDescription)) {
    // uzupelnienie pol wycieczki danymi 
    // wprowadzonymi przez uzytkownika
    newTrip.tripName = tripName
    newTrip.tripDescription = tripDescription
    newTrip.tripIsPrivate = tripIsPrivate.isChecked

    // wywolanie funkcji dodajacej miejsce do bazy danych
    // z obiektu ViewModel
    firestoreViewModel.addTripToFirestore(
        currentPlace.placeId!!, newTrip
    )
} else {
    // wyswietlenie powiadomienia w razie nie uzupelnienia 
    // przez uzytkownika wszystkich wymaganych parametrow
    Toast.makeText(
        context, 
        "Please fill all fields.", 
        Toast.LENGTH_SHORT
    ).show()
}
        \end{lstlisting}
        \vspace{1cm}