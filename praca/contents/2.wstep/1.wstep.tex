\newpage
\section{Wstęp}

\vspace{1cm}
    Projekt powstał z myślą o stworzeniu aplikacji społecznościowej pozwalającej użytkownikom na dzielenie się ciekawymi miejscami, które odwiedzili, oraz 
    pozwalającej na planowanie wycieczek. Aplikacja może posłużyć jako narzędzie pomagające turystom i obcokrajowcom poznać ciekawe obiekty w nieznanej okolicy.
    Motywacją do stworzenia aplikacji była też chęć nauczenia się tworzenia aplikacji integrujących w sobie technologie Google Maps Platform i Firebase, a także chęć stworzenia środowiska 
    społecznościowego pozwalającego na komunikowanie się niewykorzystujące pisemnego wkładu użytkowników.  

    Po zarejestrowaniu się za pomocą adresu e-mail lub konta Twitter, użytkownik witany jest przez ekran główny wyświetlający mapę oraz pasek nawigacji.
    Za pomocą znajdującego się na środku paska nawigacji przycisku ,,Dodaj'' użytkownik może w miejscu na mapie znajdującym się w centrum ekranu ustawić Pinezkę, 
    dodając tym samym Miejsce do bazy danych. Użytkownik może ustalić nazwę, opis i kategorię miejsca, oraz określić, czy jest to miejsce prywatne, czy publiczne. Korzystając z przycisków znajdujących się
    na górze ekranu, użytkownik może przeszukać okolicę widoczną na ekranie w poszukiwaniu Miejsc. Dolne menu umożliwia użytkownikowi nawigację pomiędzy widokiem mapy,
    ustawień konta użytkownika, listą ulubionych miejsc i wycieczek użytkownika, listą miejsc i wycieczek stworzonych przez użytkownika oraz listą najpopularniejszych wycieczek.
    Po przeglądnięciu okolicy w poszukiwaniu Miejsc, na mapie wyświetlą się pinezki reprezentujące Miejsca. Kliknięcie jednej z nich otworzy menu prezentujące okoliczne Miejsca wraz z ich opisem.
    W tym oknie użytkownik ma możliwość dodania miejsca do istniejącej już wycieczki lub stworzenia nowej, zawierającej wybrane miejsce. 

    Istotnym elementem projektu jest system Etykiet (Tagów). Umożliwia on wzajemną komunikację użytkowników bez konieczności pisemnego wkładu
    takiego, jakim są np.\@ komentarze. Jest to prosty system pozwalający każdemu użytkownikowi na przypisanie każdemu publicznemu Miejscu dowolnej liczby spośród predefiniowanego zbioru Etykiet.
    Każde Miejsce zlicza ilość nadanych mu przez użytkowników Etykiet i wyświetla 10 najpopularniejszych z nich w menu Miejsca. Pozwala to na pewien sposób oceny miejsca pod kątem jego przeznaczenia,
    a nawet reinterpretacji przeznaczenia założonego przez autora miejsca. 

    Aplikacja została stworzona za pomocą języka programowania Kotlin, który jest zalecanym przez firmę Google językiem do tworzenia natywnych aplikacji na system operacyjny Android.
    Jest on rozwijany przez firmę JetBrains. Firma ta jest także twórcą Android Studio --- zintegrowanego środowiska programistycznego dedykowanego do programowania aplikacji na urządzenia
    z systemem Android. Android Studio posiada zintegrowane emulatory urządzeń oraz szereg narzędzi ułatwiających tworzenie aplikacji mobilnych. 

    Wszelkiego rodzaju dane zapisywane są przez aplikację w chmurze obliczeniowej Google za pośrednictwem serwisu Firebase. Firebase to serwis typu ,,Back-end jako Usługa'' 
    (ang. \emph{Back-end as a Service}, BaaS) należący do firmy Google. Komunikacja pomiędzy Firebase a aplikacją zachodzi za pomocą specjalistycznych bibliotek, których zastosowanie zostanie
    przybliżone w dalszej części pracy. Dane użytkowników, miejsc oraz wycieczek zapisywane są w oferowanej w usługach serwisu Firebase nierelacyjnej bazie danych Firestore. 

    Google Maps Platform to obszerny zestaw wielu Interfejsów Programowania Aplikacji (ang.\@ \emph{application programming interface}, API) należących do trzech głównych kategorii:
    Maps (mapy), Places (miejsca) i Routes (trasy). W niniejszej pracy zastosowałem należące do pierwszej kategorii ,,Maps SDK for Android'' oraz należące do trzeciej
    kategorii Directions API.\@ Opis integracji tych interfejsów zawarty jest w dalszych częściach pracy.

    Kod źródłowy programu dostępny jest w repozytorium w serwisie GitHub, pod adresem https://github.com/baqterya/WroclawTripPlanner.