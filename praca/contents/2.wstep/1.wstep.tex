\newpage
\section{Wstęp}

\vspace{1cm}
    Projekt powstał z myślą o stworzeniu aplikacji społecznościowej pozwalającej użytkownikom na dzielenie się ciekawymi miejscami, które odwiedzili, oraz 
    pozwalającej na planowanie wycieczek. Aplikacja może posłużyć jako narzędzie pomagające turystom i obcokrajowcom poznać interesujące lokacje w swojej nowej okolicy.
    Motywacją do stworzenia aplikacji była chęć napisania aplikacji integrującej w sobie wykorzystanie Google Maps Platform, serwisu Firebase oraz stworzenia systemu 
    społecznościowego pozwalającego na komunikację niewykorzystującą pisemnego wkładu użytkowników, który przyjął formę Etykiet. \\ 

    Po zarejestrowaniu się za pomocą adresu e-mail lub konta Twitter użytkownik powitany jest przez ekran główny zawierający na sobie mapę oraz pasek nawigacji.
    Za pomocą znajdującego się na środku paska nawigacji przycisku „Dodaj” użytkownik może w miejscu na mapie znajdującym się w centrum ekranu ustawić Pinezkę, 
    dodając tym samym Miejsce do bazy danych. Użytkownik może ustalić nazwę, opis i kategorię miejsca, oraz jego prywatność. Korzystając z przycisków znajdujących się
    na górze ekranu, użytkownik może przeszukać okolicę widoczną na ekranie w poszukiwaniu Miejsc. Dolne menu umożliwia użytkownikowi nawigację pomiędzy widokiem mapy,
    ustawień konta użytkownika, listą ulubionych miejsc i wycieczek użytkownika, listą miejsc i wycieczek stworzonych przez użytkownika oraz listą najpopularniejszych wycieczek.
    Po zeskanowaniu okolicy w poszukiwaniu miejsc na mapie wyświetlą się pinezki reprezentujące miejsca. Kliknięcie jednej z nich otworzy menu prezentujące okoliczne miejsca i ich detale.
    Tam użytkownik ma możliwość dodania miejsca do istniejącej już wycieczki lub stworzenia nowej, zawierającej wybrane miejsce. \\ 

    Istotnym elementem projektu jest także system Etykiet (Tagów). Jest on sposobem na wprowadzenie pewnego stopnia komunikacji użytkowników bez konieczności pisemnego wkładu
    takiego, jakim są np.\@ komentarze. Jest to prosty system pozwalający każdemu użytkownikowi na dodanie wybranych z listy Etykiet do każdego publicznego miejsca w bazie danych.
    Każde miejsce zlicza ilość nadanych przez użytkowników etykiet i wyświetla 10 najpopularniejszych z nich w menu miejsca. Pozwala to na pewien sposób oceny miejsca pod kątem jego przeznaczenia,
    a nawet reinterpretacji przeznaczenia założonego przez autora miejsca. \\

    Aplikacja została stworzona za pomocą języka programowania Kotlin, który jest zalecanym przez firmę Google językiem do tworzenia natywnych aplikacji na system operacyjny Android.
    Jest on rozwijany przez firmę JetBrains. Firma ta jest także twórcą Android Studio — Zintegrowanego Środowiska Programistycznego dedykowanego do programowania aplikacji na urządzenia
    z systemem Android. Android Studio posiada zintegrowane emulatory urządzeń oraz szereg narzędzi ułatwiających tworzenie aplikacji mobilnych. \\

    Wszelkiego rodzaju dane zapisywane są przez aplikację w Chmurze Obliczeniowej Google za pośrednictwem serwisu Firebase. Firebase to serwis typu Back-end Jako Usługa 
    (ang. Back-end as a Service, BaaS) należący do firmy Google. Komunikacja pomiędzy Firebase a aplikacją zachodzi za pomocą dedykowanych bibliotek, których zastosowanie zostanie
    przybliżone w dalszej części pracy. Dane użytkowników, miejsc oraz wycieczek zapisywane są w oferowanej w usługach serwisu Firebase nierelacyjnej bazie danych Firestore. \\

    Google Maps Platform to obszerny zestaw wielu Interfejsów Programowania Aplikacji (ang.\@ application programming interface, API) należących do trzech głównych kategorii:
    Maps (mapy), Places (miejsca) i Routes (trasy). W ninejszej pracy zastosowałem należące do pierwszej kategorii „Maps SDK for Android” oraz należące do trzeciej
    kategorii Directions API.\@ Opis integracji tych interfejsów zawarty jest w dalszych częściach pracy.