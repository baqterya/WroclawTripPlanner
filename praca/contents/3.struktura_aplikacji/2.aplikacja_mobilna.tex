\newpage            
	\subsection{Aplikacja mobilna}
	Aplikacja mobilna stworzona została z myślą o działaniu na systemie operacyjnym Android. Jest to system operacyjny na urządzenia mobilne, oparty na zmodyfikowanym jądrze Linuxa.
	Jest rozwijany przez sojusz biznesowy Open Handset Alliance~\cite{OHA}. Android tworzony jest w modelu otwartego oprogramowania, finansowanego w znacznej mierze przez Google.
	Według badań prowadzonych przez firmę Statista, wraz ze styczniem 2022 Android jest najpopularniejszym systemem operacyjnym na urządzenia mobilne, posiadając prawie 70\% 
	udziału w rynku urządzeń mobilnych~\cite{OS_SHARE}. Ponadto system ten jest także wykorzystywany jako system operacyjny samochodowy (Android Auto), telewizyjny (Android TV) oraz
	na urządzenia mobilne typu Smartwatch (Wear OS). \\

		\subsubsection{Kotlin i Android Studio}
		Aplikacja mobilna została napisana w języku programowania Kotlin~\cite{KT_MAIN}. Jest to wieloplatformowy, statycznie typowany język programowania stworzony i rozwijany przez firmę JetBrains.
		Od 2019 roku jest on zalecany przez Google jako podstawowy język programowania na system operacyjny Android~\cite{ANDROID_KT}. Język ten jest zaprojektowany z myślą o
		pełnym współdziałaniu z językiem Java, poprzedniku Kotlina jako głównego języka w programowaniu na Androida. Poza językiem Kotlin wykorzystany został także język znaczników XML, służący do projektowania warstwy graficznej aplikacji, oraz Gradle, narzędzie do automatyzacji budowania projektów, 
		wspierane jako oficjalny system budowania projektów na system Android. \\

		Aplikacja została napisana, wykorzystując Zintegrowane Środowisko Deweloperskie (ang. Integrated Development Envirnoment, IDE) Android Studio~\cite{MEET_AS}, Jest to oficjalne IDE systemu operacyjnego
		Android stworzone we współpracy firm Google i JetBrains. Oferuje ono wiele narzędzi wspierających budowanie aplikacji na system Android. Android Studio zapewnia m.in.\@ inteligenty edytor kodu
		uzupełniający kod pisany w Kotlinie, Javie lub C/C++, graficzny edytor wspomagający tworzenie wizualnej części aplikacji, wbudowany, prosty w obsłudze emulator urządzenia mobilnego z 
		systemem Android, czy zintegrowany system budowania projektów Gradle. \\

\newpage
		\subsubsection{Model danych}
		W aplikacji wykorzystane są cztery klasy danych mające reprezentować obiekty umieszczane w bazie danych. \\
		Są to: 

		\noindent \newline User (Użytkownik) — jest to klasa zawierająca pseudonim użytkownika, jego unikalny identyfikator oraz adres e-mail. \\
		
		\noindent \newline Place (Miejsce) — najbardziej istotna klasa danych w aplikacji, reprezentująca miejsca dodawane na mapie przez użytkowników. Użytkownik dodający miejsce ma możliwość ustalenia
		jego Nazwy, Opisu, Kategorii (do wyboru z pięciu gotowych) oraz prywatność (prywatne miejsce nie zostanie wyświetlone na mapie podczas wyszukiwania). Ponadto, miejsce zawiera w sobie:
		unikalny identyfikator, długość i szerokość geograficzną (sczytane z położenia kursora w trakcie dodawania miejsca), GeoHash (typ danych, będący zaszyfrowanymi współrzędnymi geograficznymi miejsca, 
		mający usprawnić wydajność geograficznych zapytań do bazy danych~\cite{GEOHASH}), identyfikator i pseudonim autora miejsca, listę kategorii (ponieważ miejsce może należeć do więcej niż jednej 
		kategorii na raz), listę etykiet przypisanych miejscu przez wszystkich użytkowników, licznik polubień danego miejsca, listę identyfikatorów użytkowników, którzy polubili dane miejsce oraz flagę 
		określającą prywatność miejsca. 
		
		\noindent \newline Tag (Etykieta) — prosta klasa tworząca system etykiet. System ten inspirowany jest systemem etykiet wykorzystanym przez firmę Valve w platformie dystrybucji cyfrowej Steam.
		Jest to platforma, która łączy w sobie funkcje platformy dystrybucji programów komputerowych (głównie gier komputerowych i oprogramowania z nimi związanego) oraz serwisu społecznościowego.
		System etykiet stworzony przez Valve pozwala użytkownikom na dodawanie do znajdujących się w ofercie programów etykiet w postaci kilku słów np. „łamigłówki”, „trudna”, „gra z otwartym światem”.
		Znajdujących się w bazie danych etykiet można używać, aby filtrować wyniki wyszukiwania w ofercie platformy Steam~\cite{STEAM}. System znaczników obecny w aplikacji mobilnej podobnie jak system
		Valve pozwala każdemu użytkownikowi na dodanie etykiety do każdego publicznego miejsca. Z uwagi na rozmiar projektu i ograniczenia techniczne, system ten nie umożliwia użytkownikom tworzenia nowych
		etykiet, a jedynie na korzystanie z gotowego zbioru etykiet prezentowanego w menu dodawania etykiet. Klasa etykiety składa się z nazwy etykiety, licznika dodań, oraz listy użytkowników, którzy
		dodali etykietę do miejsca. W menu wyświetlającym szczegóły miejsca znajduje się lista wyświetlająca najpopularniejsze etykiety. Każdy użytkownik może dodać konkretną etykietę tylko raz, a 
		w bazie danych zliczane jest, ile razy dana etykieta została dodana do danego miejsca. To pozwala na wyświetlenie w menu przedstawiającym szczegóły miejsca, dziesięciu najpopularniejszych etykiet
		danego miejsca. Pozwala to użytkownikom na określanie tego, jakie aktywności można w danym miejscu wykonywać np. „muzeum”, „pomnik”, „bar”, „kino” itp. Najpopularniejsze etykiety powinny najlepiej
		reprezentować przeznaczenie danego miejsca.

		\noindent \newline Trip (Wycieczka) — jest to klasa reprezentująca zbiór miejsc, wykorzystywana do tworzenia zapytań o trasę w Directions API.\@ Użytkownik tworzący wycieczkę
		może ustalić jej nazwę, opis i prywatność, a następnie może dodawać do niej miejsca. Każda wycieczka może zawierać w sobie do 10 miejsc, z uwagi na koszt odpowiedzi na bardziej złożone zapytania
		w Directions API.\@ Poza polami ustalanymi przez użytkownika wycieczka zawiera także unikalny identyfikator, identyfikator i pseudonim autora, licznik polubień danej wycieczki oraz listę identyfikatorów 
		użytkowników, którzy polubili daną wycieczkę. \\

		\noindent Dokładniejsza reprezentacja klas danych oraz zachodzących pomiędzy nimi relacji przedstawiona jest na poniższym diagramie (patrz Rysunek~\ref{relations}).

		\vspace{1cm}
        \begin{figure}[!ht]%
            \centering
            \includegraphics[scale=0.34]{src/relations_diagram.png}
            \caption{Diagram przedstawiający relacje w bazie danych.\label{relations}}
            \qquad
        \end{figure} 
