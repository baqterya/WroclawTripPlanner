\newpage
    \section{Podsumowanie}
    Zamysłem niniejszej pracy było stworzenie narzędzia pozwalającego użytkownikom na zapisywaniu ciekawych miejsc we Wrocławiu i na dzieleniu się nimi
    z innymi użytkownikami. Głównymi celami w tworzeniu aplikacji były autoryzacja użytkowników, możliwość zapisywania miejsc oraz wycieczek do bazy danych Cloud Firestore,
    oraz zapewnienie użytkownikowi usług nawigacyjnych: wskazywania jego pozycji oraz rysowania ścieżek prowadzących do miejsc lub obrazujących przebieg wycieczki. Cele 
    te zostały zrealizowane. 

    Moja aplikacja może stanowić podstawę do rozwoju bardziej rozbudowanej aplikacji nawigacyjno-społecznościowej. Dzięki wykorzystaniu usług Firebase, zastosowana w niej baza danych jest otwarta na rozwój
    i skalowanie. Dzięki dobrej wydajności Cloud Firestore nie będzie problemem  ani ewentualny wzrost liczby użytkowników, ani ilości wprowadzanych przez nich danych.
    Co więcej, aplikacja próbuje poruszyć istotny temat w projektowaniu aplikacji społecznościowych jakim jest aktywizacja użytkowników. Zaimplementowany w niej system znaczników 
    stanowi elastyczną bazę, na której można by zbudować zaawansowany system komunikacji między użytkownikami. Znaczniki wymagają one od nich zdecydowanie mniej inicjatywy
    niż ma to miejsce w przypadku pisania tradycyjnych komentarzy. Dzięki temu można liczyć na większy udział użytkowników w wykorzystywaniu tej funkcji. 

    Proces tworzenia aplikacji nie był pozbawiony problemów. Zastosowanie Material Design pozwala na uzyskanie przejrzystego, estetycznego i responsywnego interfejsu, lecz w 
    paru miejscach ustawienie przycisków nie jest dostatecznie dobrze przemyślane. Niektóre funkcjonalności interfejsu, takie jak przesuwanie na boki miejsc w arkuszu szczegółów Miejsc,
    mogą być nieintuicyjne. W wypadku przyszłego rozwoju aplikacji nacisk powinien zostać położony na rozbudowanie interfejsu rysowania trasy wycieczki, który, chociaż funkcjonalny,
    odstaje w kwestii wygody obsługi. Rozbudowie powinien ulec także ekran ustawień, który w obecnym stanie jest dość ubogi. W przyszłym rozwoju aplikacji powinny zagościć w nim np.\ takie opcje jak
    zmiana języka aplikacji czy zaimplementowanie zmiany motywu pomiędzy jasnym a ciemnym. 

    Mimo wszystko tworzenie aplikacji odbyło się bez większych problemów technicznych. Zawdzięczam to wygodzie używania chmury obliczeniowej Firebase oraz oferowanym przez Google
    bibliotekom i narzędziom. Istotne jest też to, że wykorzystane w niniejszej pracy narzędzia są narzędziami współczesnymi i wciąż na bieżąco rozwijanymi. Dzięki temu dostępna jest obszerna i przystępna dokumentacja.
    Proces budowy aplikacji dowodzi, że Firebase jest potężnym i przystępnym w obsłudze narzędziem, dzięki któremu nawet jednoosobowy zespół jest w stanie w krótkim czasie wdrożyć aplikację
    wykorzystującą rozbudowaną bazę danych i szereg bezpiecznych i wydajnych rozwiązań sieciowych takich jak autoryzacja użytkowników.