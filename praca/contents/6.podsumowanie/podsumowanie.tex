\newpage
    \section{Podsumowanie}
    Zamysłem niniejszej pracy było stworzenie narzędzia pozwalającego użytkownikom na zapisywaniu ciekawych miejsc we Wrocławiu i na dzieleniu się nimi
    z innymi użytkownikami. Głównymi celami w tworzeniu aplikacji były autoryzacja użytkowników, możliwość zapisywania miejsc oraz wycieczek do bazy danych Cloud Firestore,
    oraz zapewnienie użytkownikowi usług nawigacyjnych: wskazywania jego pozycji, oraz rysowania ścieżek prowadzących do miejsc lub obrazujących przebieg wycieczki. Cele 
    te zostały zrealizowane. \\ 

    Aplikacja może stanowić podstawę do rozwoju rozbudowanej aplikacji nawigacyjno-społecznościowej. W związku z wykorzystaniem usług Firebase jest baza danych otwarta, jest na rozwój
    i skalowanie. Dzięki wydajności Cloud Firestore nie będzie przeszkodą wzrost liczby użytkowników ani wprowadzanych przez nich do bazy danych Cloud Firestore danych.
    Ponadto, aplikacja próbuje poruszyć istotny temat w projektowaniu aplikacji społecznościowych, jakim jest aktywizacja użytkowników. System znaczników zaimplementowany
    w aplikacji stanowi elastyczną podstawę na bazie, której zbudować można by zaawansowany system komunikacji użytkowników. Wymagają one od nich zdecydowanie mniej inicjatywy
    niż pisanie tradycyjnych komentarzy, dzięki czemu można liczyć na większy udział użytkowników w używaniu tej funkcji. \\

    Proces tworzenia aplikacji nie odbył się bez problemów. Zastosowanie Material Design pozwala na uzyskanie przejrzystego, estetycznego i responsywnego interfejsu, lecz w 
    paru miejscach ustawienie przycisków nie jest dostatecznie dobrze przemyślane. Niektóre funkcjonalności interfejsu, takie jak przesuwanie na boki miejsc w arkuszu szczegółów miejsc
    mogą być nieintuicyjne. W wypadku przyszłego rozwoju aplikacji nacisk powinien zostać położony na rozbudowanie interfejsu rysowania trasy wycieczki, który, chociaż funkcjonalny,
    odstaje w kwestii wygody obsługi. Rozbudowie powinien ulec także ekran ustawień, który w obecnym stanie jest dość ubogi. W przyszłym rozwoju aplikacji powinny zagościć w nim np.\@ takie opcje jak
    zmiana języka aplikacji, czy zaimplementowanie zmiany motywu pomiędzy jasnym a ciemnym. \\

    Mimo wszystko tworzenie aplikacji odbyło się bez większych problemów technicznych. Można zawdzięczać to wygodzie w używaniu chmury obliczeniowej Firebase i oferowanych przez Google
    bibliotekom i narzędziom. Istotny jest też fakt tego, że wykorzystane w pracy narzędzia są narzędziami współczesnymi i wciąż rozwijanymi. Dzięki temu dostępna jest obszerna i przystępna dokumentacja.
    Proces budowy aplikacji dowodzi, że Firebase jest potężnym i przystępnym w obsłudze narzędziem, dzięki któremu nawet jednoosobowy zespół jest w stanie w krótkim czasie wdrożyć aplikację
    wykorzystującą rozbudowaną bazę danych i szereg bezpiecznych i wydajnych rozwiązań sieciowych takich jak autoryzacja użytkowników.